%%%%%%%%%%%%%%%%%%%%%%%%%%%%%%%%%%%%%%%%%
% Medium Length Professional CV
% LaTeX Template
% Version 2.0 (8/5/13)
%
% This template has been downloaded from:
% http://www.LaTeXTemplates.com
%
% Original author:
% Thanks : Rishi Shah 's Contribution
% inspired by his awesome contribution:
% https://www.overleaf.com/articles/rishi-shahs-resume/vgxvkmxktyxn
% Author : Allianzcortex
% contact me : github.com/Allianzcortex
% email : iamwanghz#gmail.com
%
% Important note:
% This template requires the resume.cls file to be in the same directory as the
% .tex file. The resume.cls file provides the resume style used for structuring the
% document.
%
%%%%%%%%%%%%%%%%%%%%%%%%%%%%%%%%%%%%%%%%%

%----------------------------------------------------------------------------------------
%	PACKAGES AND OTHER DOCUMENT CONFIGURATIONS
%----------------------------------------------------------------------------------------

\documentclass{resume} % Use the custom resume.cls style

\usepackage[left=0.40in,top=0.3in,right=0.75in,bottom=0.1in]{geometry} % Document margins
\usepackage{fontawesome}
\usepackage{times}
\newcommand{\tab}[1]{\hspace{.2667\textwidth}\rlap{#1}}
\newcommand{\itab}[1]{\hspace{0em}\rlap{#1}}
% \begin{center}
% {\centerline {\em \textbf {Seeking for a fulltime internship from Sep 2019 - Apr 2010(8 months) } } }
% \end{center}
\name{Ethan Nagelvoort} % Your name 


%\address{123 Pleasant Lane \\ City, State 12345} % Your secondary addess (optional)
\address{ 9362 Shackleford Ct., San Diego, CA 92126}
\address{(858)-610-8778} 
\address{ {github.com/Ethan-Nagelvoort} \hspace*{0.5em}\hspace*{0.5em}{ linkedin.com/in/ethan-nagelvoort} \hspace*{0.5em}\hspace*{0.5em}{ ethannagelvoort@gmail.com}}

\begin{document}
%----------------------------------------------------------------------------------------
%	EDUCATION SECTION
%----------------------------------------------------------------------------------------

\begin{rSection}{Education}

{\bf San Diego State University } \hfill {\em August 2017 - Present} 
\\{ \text {Pursuing a B.S. in Computer Engineering }} \hfill {\em GPA: 3.52} 
\\{ \text {Expected Graduation: December 2021}}

\end{rSection}

\begin{rSection}{Technical Skills}

\textbf {Programming Languages:} &\text{  C, C\#, C++, HTML, MIPS, Verilog, Kotlin, ARM, MATLAB, R, Java, Python} 

\textbf{Frameworks/Tools:} Xilinx Vivado, Arduino IDE, Visual Studio, Android Studio, Atmel Studio, Eclipse, QtSpim, LTSpice, PuTTY, Multisim, Ubuntu, Linux, NI software, Cadence, MySQL


\end{rSection}

% \begin{rSection}{Carrier Objective}
%  To work for an organization which provides me the opportunity to improve my skills and knowledge to grow along with the organization objective.
% \end{rSection}
%--------------------------------------------------------------------------------
%    Projects And Seminars
%-----------------------------------------------------------------------------------------------
\begin{rSection}{Projects Experience (All available in Github)}

{\bf Embedded Operating Systems Programming: Scheduling Algorithms (C, Linux)
}
\\- Implemented RR, SJF, FCFS, and MLFQ scheduling algorithms to simulate how they would divide up four different \hspace*{0.5em}processes created through the fork() function.
\\- Used timestamps to measure run time for each process and overall context switch overhead.

{\bf Microprocessor Programming: Single Cycle, Multicycle, and Pipeline Processors (Verilog, Vivado)}
\\- Created these three different types of MIPS processors that utilize many parametric modules such as multiplexers, \hspace*{0.5em}ALU, register file, instruction memory, and data memory.
\\- Designed to run MIPS programs that include R-type, J-type, and I-type instructions to compute numerical values.

{\bf Microcontroller Programming: Keycode Lock (C++, Arduino IDE) }
\\- Collaborated with a team of 4 students to design a system that lights a red LED or a green LED depending on if the \hspace*{0.5em}correct 4 digit keycode is entered on a keypad.
\\- Utilized an Arduino Uno to enable a LM339 op-amp to drive the LEDs by switching logic 1 or 0.   




\end{rSection}
%----------------------------------------------------------------------------------------
%	TECHNICAL STRENGTHS SECTION
%----------------------------------------------------------------------------------------


%----------------------------------------------------------------------------------------
%	WORK EXPERIENCE SECTION
%----------------------------------------------------------------------------------------
\begin{rSection}{Work Experience}
{\bf San Diego State University, ECE Department  } \hfill {\em September 2021 - Present} 
\\Teacher’s Assistant for Electrical Engineering 210: Circuit Analysis 1
\\- Assisted the professor in grading coursework, structuring the class, and proctoring exams.
\\-  Facilitated labs that utilized myDAQ technology, NI software, and Multisim software to build and simulate       \\\hspace*{0.5em}series/parallel resistor circuits, RC circuits, and voltage/current divider circuits.  

{\bf GAOTek Inc.   } \hfill {\em July 2021 - October 2021} 
\\Software Developer Assistant/Intern
\\- Operated as a front-end app developer through the use of Kotlin and Android Studios.
\\- Worked with a team of app developers to help create an app store that incorporates many of GAOTek’s other apps.
\\- Received three certifications that exemplify one's success and hard work at this internship.

{\bf San Diego State University, Supplemental Instruction   } \hfill {\em January 2020 - Present} 
\\SI leader for Electrical Engineering 210: Circuit Analysis 1
\\- Facilitated interactive activities for students to partake in as practice for course material.
\\- Helped students understand entry level concepts such as Thevenin and Norton equivalence.
\\- Participated in weekly meetings to discuss and improve upon my skills as an SI leader.       


% \\- xx
% \\- xx
% there should be more lines

%Minor in Linguistics \smallskip \\
%Member of Eta Kappa Nu \\
%Member of Upsilon Pi Epsilon \\


\end{rSection}
\end{document}
